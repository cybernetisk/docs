\documentclass[10pt,norsk,a4paper]{article}
\usepackage[utf8]{inputenc}
\usepackage[T1]{fontenc}
\usepackage[norsk]{babel}
\usepackage[cm]{fullpage}
\usepackage{color}
\usepackage{parskip,textcomp,amssymb,graphicx}

\title{Generalforsamling\\
	Våren 2015\\
	Cybernetisk Selskab\\[.3cm]
	\includegraphics[width=0.08\textwidth]{./Images/cyb-seal.png}\\[-.5cm]}
\date{21. mai 2015}

\begin{document}
\maketitle{}
\tableofcontents{}

%\newpage
~\\

\section{Valg av møteleder}
Ole Kristian Rosvold\\
Valgt ved akklamasjon.

\section{Valg av referent}
Marte Pedersen\\
Valgt ved akklamasjon.

\section{Valg av protokollunderskrivere}
Dan-Mikkel Hübenette\\
Torgeir Lebesbye\\
Valgt ved akklamasjon.

\section{Valg av tellekorps}
Aleksi Luukkonen\\
Vegard Lillevoll\\
Valgt ved akklamasjon.

\section{Godkjenning av innkalling}
Godkjent ved akklamasjon.

\section{Godkjenning av dagsorden}
Godkjent ved akklamasjon.

\newpage


\section{Semesterberetninger}
\subsection{Semesterberetning ved Leder}
Nå som vi står ved slutten av et nytt semester er det på tide å ta en oppsummering av hva som har skjedd i foreninga de siste 6 månedene.\\

Som tidligere er vår største utfordring at vi har en god del færre frivillige enn vi skulle ønske, men med iherdig innsats fra alle i foreninga har vi også i år økt aktivitetsnivået noe.\\

I tillegg til våre faste arrangementer har vi dette semesteret videreført samarbeidet med INF1000 slik at det har vært et tilbud til INF1010-studenter. Vi har også avholdt arrangementer sammen med både Navet og dagen@ifi.\\

Dette faller godt inn under målet om å styrke samarbeidet internt på Ifi og det er med glede jeg kan si at dette går rett vei. I år som i fjor ligger vi veldig solid an økonomisk, hvilket gir oss gode muligheter til å gjennomføre større prosjekter om det skulle bli aktuelt. \\

Til slutt vil jeg takke alle som har vært med å jobbe dette semesteret. Ifi ville vært en mye kjedeligere plass uten dere.\\
\

Jan Kristian Furulund,\\
Leder Cybernetisk Selskab\\
Vår 2015

%\newpage


\subsection{Semesterberetning ved Kjellermogul}
Dette semesteret har gått ganske bra. Vi har hatt mye salg i baren, og har fått utvidet utvalget i baren. For ikke lenge siden fikk vi satt in flere øltapper i baren, som er noe vi har snakket om i flere år.\\

Vi har hatt litt mangel på frivillige i baren, som har gjort det litt vanskelig på gjennomføre noen arrangementer, men planer har blitt lagt for å rekruttere mye neste semester.\\

Noe som er nytt for semesteret er hvor mange utvekslingstudenter vi har hatt i foreningen. De har spesiellt hjulpet veldig mye til i baren og laget veldig god stemning. Vi har måttet lært oss å bli flinkere til å komme med informasjon på engelsk, men ellers har det fungert veldig bra. Det er trist at de snart må dra igjen.\\
\

Atle Nordland,\\
Kjellermogul Cybernetisk Selskab\\
Vår 2015

\subsection{Spørsmål}
Torgeir Lebesbye lurer på hvor mange frivillige man har hatt dette semesteret. Jan Kristian Furulund: Vi har ikke et veldig godt tall på det, rundt 100.

\newpage


\section{Regnskap 2013}
Vil ikke bruke så mye tid på 2013.\\
Ikke ønske fra GF om å gå gjennom dette.


\section{Regnskap 2014}
Våren:
Asgeir Mortensrud har spørsmål om hvorfor det er så lite varesvinn. Henrik Steen: Oversikten er ikke så god, vi har ikke oversikt over hva som går ut med fat osv., svinn føres, men det går ikke helt inn i regnskapet.\\

Høsten:
Martin Evensen: Hvilke arbeidsgrupper går under her. Henrik Steen: Internansvarlig har ansvar for dette. Vegard Lillevoll: Web, arr, dj, kafé, blæst, og økonomi. Henrik: Hvis det er andre grupperinger som burde inn, så kan vi se på det.\\
Utlån: Vi får penger tilbake fra SPF senere, kanskje juni.\\

Godkjent ved akklamasjon.

%\newpage


\section{Revidert budsjett 2015}
Henrik Steen må orientere om balanseregnskapet fra tidligere år.\\

Våren:\\
Ingen kommentarer.\\

Høsten:\\
Asgeir Mortensrud håper man tar ansvar for å bruke opp de 4 000 kr som er satt opp for denne GF-en.\\

Martin Evensen bemerker at brettspill er en investering til Escape, ikke noe ala. potetgull og brus, kan kanskje settes under drift?\\

Budsjett godkjent ved akklamasjon.

%\newpage

\section{Kontingentfastsettelse}
Hovedstyret foreslår å holde medlemskontigenten på 30 kroner.\\

Godkjent ved akklamasjon.

\section{Valg}
Blir tatt på engelsk. It is going very bad.\\
\begin{minipage}[t]{9cm}
\subsection{Hovedstyret}
Man velges inn i hovedstyret for ett år av gangen.
\subsubsection{Leder}
Jan Furulund stiller til gjenvalg.\\
Valgt ved akklamasjon.
\subsubsection{Nestleder}
Odd-Tørres Lunde stiller.\\
Valgt ved akklamasjon.
\subsubsection{Kjellermogul}
Nikolas Papaioannou stiller.\\
Valgt ved akklamasjon.
\end{minipage}
\begin{minipage}[t]{9cm}
\subsection{Kjellerstyret}
Økonomiansvarlig velges for ett år av gangen, mens de resterende vervene velges for ett semester.
\subsubsection{Barsjef}
Morgaine Wood stiller.\\
Valgt ved akklamasjon.
\subsubsection{Innkjøpsansvarlig}
Fredrik Hov stiller til gjenvalg.\\
Valgt ved akklamasjon.
\subsubsection{Kafésjef}
Mette Sundal stiller.\\
Valgt ved akklamasjon.
\subsubsection{Teknisk ansvarlig}
Ingen som stiller, vil etterfylles.
\subsubsection{DJ-sjef}
Judee Isabel Bjørgan stiller.\\
Valgt ved akklamasjon.
\subsubsection{Utlånsansvarlig}
Christian Resell stiller.\\
Valgt ved akklamasjon.
\end{minipage}

%\newpage

\section{Vedtektsendringer}
Det kom ikke inn noen vedtektsendringer i år.

\section{Æresmedlemskap}

\section{Utdeling av pins}
Mats Astrup Scjølberg deler ut til:\\
Nikolas Papaiannaou\\
Kaisa Korsak (ikke til stede)\\
Gøran Frost\\
Kaja Stene (ikke til stede)\\
Martin Evensen\\

\newpage

\section*{Vedlegg: Vedtekter}
\textit{Oppdatert etter generalforsamlingen 22.05.2014.}\\

\section*{§1 Tilhørighet og formål.}
\begin{enumerate}
	\item{Cybernetisk Selskab er instituttforening for Institutt for informatikk ved Universitetet i Oslo.}
	\item{Cybernetisk Selskabs formål er å arrangere og fremme aktiviteter og arrangementer for studenter ved instituttet. Gjennom dette ønsker man å styrke miljøet og skape kameratslig samvær og faglig innhold ved siden av studiene.}
\end{enumerate}

\section*{§2 Medlemskap.}
\begin{enumerate}
	\item{Cybernetisk Selskab har tre kategorier medlemmer:}
	\begin{itemize}
		\item{Semesterbetalende medlemmer.}
		\item{Livsvarige medlemmer.}
		\item{Æresmedlemmer.}
	\end{itemize}
	\item{Kontingent for livsvarlig medlemskap er ti ganger ordinær semesterkontingent.}
	\item{Æresmedlemmer utnevnes på generalforsamling med 2/3 flertall.}
\end{enumerate}

\section*{§3 Drift og arbeidsgrupper.}
\begin{enumerate}
	\item{Foreningens daglige drift ivaretas av arbeidsgrupper med hvert sitt ansvarsområde. Arbeidsgruppene skal i størst mulig grad handle fritt og selvstendig innenfor de rammer som er satt av hovedstyret.}
	\item{Hovedstyret har rett til innsyn i understyrer og arbeidsgruppers bruk av midler.}
	\item{Understyrer og arbeidsgrupper plikter å rapportere regnskapsrelevant informasjon til kasserer og overholde vedtatte budsjetter.}
	\item{Om nødvendig kapital til foreningens drift (100.000 kroner) samt bardrift (400.000 kroner) og vedlikehold (250.000 kroner) er oppspart skal et eventuelt driftoverskudd overføres Fordelingsutvalget ved årsskifte.}
\end{enumerate}

\section*{§4 Revisjon.}
\begin{enumerate}
	\item{Leder i samarbeid med kasserer har som oppgave å engasjere en uavhengig tredjepart til å revidere regnskap for hvert hele år.}
	\item{Etter at revideringsrapporten er ferdigstilt, skal den presenteres på første mulige generalforsamlig.}
\end{enumerate}

\section*{§5 Hovedstyret.}
\begin{enumerate}
	\item{Hovedstyret skal drive Cybernetisk Selskabs virksomhet, og sikre aktivitet og gjennomføring av foreningens formål.}
	\item{Hovedstyret i Cybernetisk Selskab skal fungere som et sentralt hovedstyre for sine understyrer og arbeidsgrupper.}
	\item{Hovedstyret skal bestå av fem til ti personer, hvor vervene leder, nestleder, kasserer, kjellermogul og arrangementssjef er faste. Øvrige verv defineres av hovedstyret før generalforsamling.}
	\item{Hovedstyret konstituerer seg i samråd med lederen på semesterets første hovedstyremøte.}
        \item{Hovedstyret kan med tre fjerdedels flertall oppnevne hovedstyremedlemmer ved behov for etterfylling av alle verv foruten leder. Den etterfylte sitter frem til neste generalforsamling. }
	\item{Alle medlemmer i henhold til §2 har møterett på hovedstyrets møter, men kun styremedlemmer har stemmerett.}
\end{enumerate}

\section*{§6 Kjellerstyret.}
\begin{enumerate}
	\item{Kjellerstyret har ansvaret for den daglige driften av kjelleren i Ole-Johan Dahls hus.}
	\item{Kjellerstyret skal ha fra 5 til 10 personer, og ledes av kjellermogul med barsjef som stedfortreder.}
	\item{Faste verv i kjellerstyret er kjellermogul, barsjef og økonomiansvarlig. Øvrige verv defineres av kjellerstyret før generalforsamling.}
	\item{Kjellerstyret har ansvar for å rapportere regnskapsrelevant informasjon til kasserer og overholde vedtatte budsjetter.}
	\item{Hovedstyret kan oppnevne kjellerstyremedlemmer ved behov for etterfylling av alle verv foruten kjellermogul.}
\end{enumerate}

\section*{§7 Generalforsamling.}
\begin{enumerate}
	\item{Generalforsamlingen er foreningens høyeste myndighet. Den er beslutningsdyktig når minst 10 prosent eller 30 stykker av medlemmene er tilstedet.}
	\item{Hovedstyret innkaller til generalforsamling. Innkallingen skal komme minimum to uker før generalforsamlingen finner sted for ordinær generalforsamling. Minimum en uke før for ekstraordinær generalforsamling. Foreløpig dagsorden offentliggjøres minst én uke i forkant til ordinær generalforsamling. For ekstraordinær generalforsamling minst tre dager i forveien.}
	\item{Forslag om vedtektsendring og andre saker som søkes tatt opp på generalforsamling må være hovedstyret i hende senest 48 timer i forveien; for ekstraordinær generalforsamling 24 timer i forveien.}
	\item{Generalforsamlingen kan foreta endringer i rekkefølgen av punktene i det endelige forslag til dagsorden. Den kan også utelukke ett eller flere av de foreslåtte punkter, så lenge dette ikke strider mot §9 og §10. Den endelige dagsorden godkjennes av generalforsamlingen.}
	\item{Ethvert medlem kan på generalforsamlingen foreslå tatt opp saker utenom den oppsatte dagsorden. Generalforsamlingen kan ikke fatte vedtak i slike saker.}
	\item{Avstemningen på medlemsmøter og generalforsamlinger skal være skriftlig når minst tre av de stemmeberettigede krever det.}
        \item{Stemmerett har alle som er medlem i henhold til §2 minst èn uke før generalforsamling.}
	\item{Ordinær generalforsamling avholdes i slutten av hvert semester. Ekstraordinær generalforsamling avholdes når hovedstyret, eller minst 1/10 av medlemmene ønsker det.}
	\item{På ordinær generalforsamling skal følgende behandles:}
	\begin{itemize}
		\item{Semesterberetning.}
		\item{Regnskap.}
		\item{Budsjett.}
		\item{Kontingentfastsettelse.}
		\item{Valg.}
	\end{itemize}
	\item{Forslag til vedtektsendring skal behandles på generalforsamling, og må få 2/3 flertall for å bli vedtatt.}
\end{enumerate}

\section*{§8 Valg.}
\begin{enumerate}
	\item{Valgbare til alle styrer er alle medlemmer i henhold til §2 og informatikkstudenter ved Universitetet i Oslo. Alle valg avgjøres ved simpelt flertall.}
	\item{Hvert hovedstyremedlem velges for to semestre av gangen. Valg foregår på generalforsamling.}
	\item{Alle hovedstyrets medlemmer velges særskilt på generalforsamling.}
	\item{Alle kjellerstyrets medlemmer velges særskilt på generalforsamling. Verv utover kjellermogul og økonomiansvarlig velges for ett semester av gangen. Kjellermogul og økonomiansvarlig velges for to semestre av gangen. }
\end{enumerate}

\section*{§9 Mistillit.}
\begin{enumerate}
    \item{Foreningens medlemmer i henhold til §2 kan fremme mistillitsforslag mot styremedlemmer og andre tillitsvalgte som er valgt i henhold til §8.}
    \item{Slike forslag skal behandles på generalforsamling, og må være fremmet minst 48 timer i forveien.}
    \item{Mistillitforslag vedtas med to tredels flertall.}
    \item{Dersom et mistillitsforslag blir vedtatt kan generalforsamlingen vedta å holde nyvalg for vervet med periode frem til neste ordinære generalforsamling.}
\end{enumerate}

\section*{§10 Oppløsning}
\begin{enumerate}
        \item{Det kreves to tredels flertall på to påfølgende generalforsamlinger for å oppløse foreningen.}
        \item{Hvis en generalforsamling vedtar å oppløse foreningen skal hovedstyret tidligst tre uker og senest fire måneder senere innkalle til ekstraordinær generalforsamling hvor saken skal behandles på nytt. Innkallingen skal komme minimum to uker før generalforsamlingen finner sted.}
        \item{Dersom foreningen oppløses, går foreningens aktiva til Fordelingsutvalget (FU) ved Institutt for informatikk (IFI).}
\end{enumerate}
\end{document}
